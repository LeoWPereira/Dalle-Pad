%%
%% Apendice.tex
%% Projeto Oficinas de Integração 3
%% Created by Leonardo Winter Pereira and Lucas Zimmermann Cordeiro on 10.03.2016
%% Copyright (C). All rights reserved
%%

%% Apêndice é um texto ou documento elaborado pelo autor do TC, ou seja,
%% se foi necessário você criar uma entrevista, um relatório, ou qualquer
%% documento com o escopo de complementar sua argumentação


%% ---------------- 4 -------------------
%% O Gerenciamento da Integração do Projeto descreve os processos necessários para assegurar que os diversos elementos do projeto sejam adequadamente coordenados.
%% A integração envolve tomada de decisão e escolhas diretamente ligadas aos objetivos do projeto e aos processos das etapas de desenvolvimento e execução do plano do projeto, assim como ao processo de controle de alterações.
%% O gerenciamento da integração é composto pelos processos: desenvolvimento do plano do projeto, execução do plano do projeto e controle integrado de mudanças
\chapter{Gerenciamento de Integração do Projeto}



    %% ---------------- 4.1 -------------------
    \section{Termo de Abertura do Projeto}

    \includepdf[pages={-}]{Documentos/Termo_de_Abertura_de_Projeto.pdf}

    %% ---------------- 4.2 -------------------
    \section{Declaração do Escopo Preliminar do Projeto}



    %% ---------------- 4.3 -------------------
    \section{Plano de Gerenciamento do Projeto}



%% ---------------- 5 -------------------
%% O Gerenciamento do Escopo do Projeto descreve os processos necessários para assegurar que o projeto contemple todo o trabalho requerido, e nada mais que o trabalho requerido, para completar o projeto com sucesso.
%% A preocupação fundamental é definir e controlar o que está ou não, incluído no projeto.
%% Ele é composto pelos processos: iniciação, planejamento do escopo, detalhamento do escopo, verificação do escopo e controle de mudanças do escopo.
\chapter{Gerenciamento de Escopo do Projeto}



    %% ---------------- 5.1 -------------------
    \section{Planejamento do Escopo}



    %% ---------------- 5.2 -------------------
    \section{Definição do Escopo}



    %% ---------------- 5.3 -------------------
    \section{Estrutura Analítica do Projeto}



%% ---------------- 6 -------------------
%% O Gerenciamento do Tempo do Projeto descreve os processos necessários para assegurar que o projeto termine dentro do prazo previsto.
%% Ele é composto pelos processos: definição das atividades, sequenciamento das atividades, estimativa da duração das atividades, desenvolvimento do cronograma e controle do cronograma.
\chapter{Gerenciamento do Tempo do Projeto}



    %% ---------------- 6.1 -------------------
    \section{Definição da Atividade}



    %% ---------------- 6.2 -------------------
    \section{Sequenciamento de Atividades}



    %% ---------------- 6.3 -------------------
    \section{Estimativa de recursos de atividade}



    %% ---------------- 6.4 -------------------
    \section{Estimativa de duração de atividade}



    %% ---------------- 6.5 -------------------
    \section{Desenvolvimento do Cronograma}



%% ---------------- 7 -------------------
%% O Gerenciamento do Custo do Projeto descreve os processos necessários para assegurar que o projeto termine dentro do orçamento aprovado.
%% Ele é composto pelos processos: planejamento dos recursos, estimativa dos custos, orçamento dos custos e controle dos custos.
%% No projeto, várias atividades afetam os custos do projeto e desta forma, o planejamento e controle dos custos são fundamentais.
\chapter{Gerenciamento do Custo do Projeto}



    %% ---------------- 7.1 -------------------
    \section{Estimativa de custos}



    %% ---------------- 7.2 -------------------
    \section{Orçamentação}



%% ---------------- 8 -------------------
%% O Gerenciamento da Qualidade do Projeto descreve os processos necessários para assegurar que as necessidades que originaram o desenvolvimento do projeto serão satisfeitas.
%% O projeto tem qualidade quando é concluído de acordo com os requisitos, especificações (o projeto deve produzir o que foi definido) e adequação ao uso (deve satisfazer às reais necessidades dos clientes).
%% O gerenciamento da qualidade é composto pelos processos: planejamento da qualidade, garantia da qualidade e controle da qualidade.
\chapter{Gerenciamento da Qualidade do Projeto}



    %% ---------------- 8.1 -------------------
    \section{Planejamento da qualidade}



%% ---------------- 9 -------------------
%% O Gerenciamento dos Recursos Humanos do Projeto descreve os processos necessários para proporcionar a melhor utilização das pessoas envolvidas no projeto. Embora seja uma área de conhecimento, na maioria das vezes, complexa e subjetiva exige constante pesquisa, sensibilidade e muita vivência do dia-a-dia para saber lidar com o ser humano.
%% É composta pelos processos: planejamento organizacional, montagem da equipe e desenvolvimento da equipe.
\chapter{Gerenciamento de Recursos Humanos do Projeto}



    %% ---------------- 9.1 -------------------
    \section{Planejamento de recursos humanos}



%% ---------------- 10 -------------------
%% O Gerenciamento das Comunicações do Projeto descreve os processos necessários para assegurar a geração, captura, distribuição, armazenamento e pronta apresentação das informações do projeto para que sejam feitas de forma adequada e no tempo certo.
%% A gestão da comunicação é frequentemente ignorada pelos gerentes de projeto, no entanto nos projetos concluídos com sucesso o gerente gasta 90% do seu tempo envolvido com algum tipo de comunicação (formal, informal, verbal, escrita).
%% Este gerenciamento é composto pelos processos: planejamento das comunicações, distribuição das informações, relato de desempenho e encerramento administrativo
\chapter{Gerenciamento de Comunicação do Projeto}



    %% ---------------- 10.1 -------------------
    \section{Planejamento das comunicações}



%% ---------------- 11 -------------------
%% O Gerenciamento dos Riscos do Projeto descreve os processos que dizem respeito à identificação, análise e resposta aos riscos do projeto.
%% A prática deste gerenciamento não é ainda muito comum na maioria das organizações e alguns autores citam que gerenciar projetos é gerenciar riscos.
%% O gerenciamento de riscos é muito importante para o sucesso do projeto e é composto pelos seguintes processos: Planejamento da Gerência de Risco, identificação dos riscos, análise qualitativa de riscos, análise quantitativa de riscos, desenvolvimento das respostas aos riscos e controle e monitoração de riscos.
\chapter{Gerenciamento do Risco do Projeto}



    %% ---------------- 11.1 -------------------
    \section{Planejamento do gerenciamento de riscos}



    %% ---------------- 11.2 -------------------
    \section{Identificação de riscos}



    %% ---------------- 11.3 -------------------
    \section{Análise Qualitativa de riscos}



    %% ---------------- 11.4 -------------------
    \section{Planejamento de respostas a riscos}



%% ---------------- 12 -------------------
%% O Gerenciamento das Aquisições do Projeto descreve os processos necessários para a aquisição de mercadorias e serviços fora da organização que desenvolve o projeto.
%% Este gerenciamento é discutido do ponto de vista do comprador na relação comprador-fornecedor.
%% Ele é composto pelos processos: planejamento das aquisições, preparação das aquisições, obtenção de propostas, seleção de fornecedores, administração dos contratos e encerramento do contrato.
\chapter{Gerenciamento das Aquisições do Projeto}



    %% ---------------- 12.1 -------------------
    \section{Planejar compras e aquisições}



    %% ---------------- 12.2 -------------------
    \section{Planejar contratações}



%% Atas de Reunião são importantes pois definem exatamente o que foi / será discutido em cada reunião da Equipe Dalle Pad!
\chapter{Atas de Reunião}


